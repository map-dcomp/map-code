\documentclass[12pt]{article}
% OR
% \documentclass[letterpaper,11pt]{report}
\usepackage{fullpage}
\usepackage{float}
\usepackage{alltt}
\usepackage{graphicx}
\usepackage{amsmath}
\usepackage[table,usenames,dvipsnames]{xcolor}

% handle characters reasonably, especially <>
\usepackage[T1]{fontenc}
\usepackage{lmodern}

% nice for tables
\usepackage{tabularx}

% FloatBarrier
\usepackage{placeins}

% html links
\usepackage{hyperref}
\hypersetup{colorlinks=true, linkcolor=blue, filecolor=blue, urlcolor=blue}
\renewcommand*{\subsubsectionautorefname}{section}
\renewcommand*{\subsectionautorefname}{section}

% algorithm environment for pseudocode 
\usepackage{algorithm}
\usepackage{algorithmic}
%\floatname{algorithm}{}

% draft - include comments as footnotes and marginpars
\newif\ifdraft
\drafttrue
\ifdraft
\typeout{DRAFT - WITH COMMENTS}
\newcommand{\doccomment}[3]%
{\marginpar{\textcolor{#2}{\bf #1}}%
%\footnote{\textcolor{#2}{#3}}%
\footnote{{\color{#2}#3}}%
}
\else
\typeout{NOT DRAFT - NO COMMENTS}
\newcommand{\doccomment}[3]{}
\fi

% comments for individuals
\newcommand{\jpscomment}[1]%
{\doccomment{SCHEWE}{Bittersweet}{#1}}


\title{RLG Plans}
\author{Jon Schewe}

\begin{document}

\maketitle

\section{Overview}
\label{sec:overview}
This document describes some aspects of the RLG plans as
\texttt{LoadBalancerPlan} objects.

The plan specifies the desired state of the network that MAP is
controlling.
The MAP agent will do it's best to make the network match the specified
plan.
The RLG can see if services are started by looking at the
\texttt{ResourceReport} and \texttt{RegionNodeState} objects.
It is expected that the RLG will pay attention to the capacity limits and
not try to start more services than are allowed on an NCP.

The MAP Agent will use the \texttt{service plan} property to determine how
many containers should be running.  Containers will only be stopped when
specified in the property \texttt{stop containers}.  If there are more
services running on an NCP than specified in \texttt{service plan} and no
containers are listed in \texttt{stop containers}, then the result will be
that extra containers will be running.

\section{Simple plan}
\label{sec:simplePlan}

\subsection{service plan}
This plan specifies that there should be 2 instances of serviceA running
\texttt{nodeA0} and 1 instance of serviceB running on nodeA1. The MAP agent will
start services as needed to make this state happen. 

\begin{itemize}
\item serviceA
  \begin{itemize}
  \item nodeA0 -> 2
  \end{itemize}

\item serviceB
  \begin{itemize}
  \item nodeA1 -> 1
  \end{itemize}

\end{itemize}

\section{Stop traffic}

This shows how to stop traffic to a container, but keep it running.
We will assume that the previous plan was the one from \autoref{sec:simplePlan} and that the MAP agent started services as follows
\begin{itemize}
\item nodeA0c0 -> serviceA
\item nodeA0c1 -> serviceA
\item nodeA1c0 -> serviceB
\end{itemize}

Assume that we want to stop traffic to container \texttt{nodeA0c1}.

\subsection{service plan}

\begin{itemize}
\item serviceA
  \begin{itemize}
  \item nodeA0 -> 2
  \end{itemize}

\item serviceB
  \begin{itemize}
  \item nodeA1 -> 1
  \end{itemize}

\end{itemize}

\subsection{stop traffic to}
\begin{itemize}
\item nodeA0c1
\end{itemize}


\section{Stop container}

This shows how to stop a container. 
We will assume that the previous plan was the one from \autoref{sec:simplePlan} and that the MAP agent started services as follows
\begin{itemize}
\item nodeA0c0 -> serviceA
\item nodeA0c1 -> serviceA
\item nodeA1c0 -> serviceB
\end{itemize}

Assume that we want to stop container \texttt{nodeA0c1}.
We need to both reduce the number of services running on \texttt{nodeA0}
and add \texttt{nodeA0c1} to the \texttt{stop containers} property.

\subsection{service plan}

\begin{itemize}
\item serviceA
  \begin{itemize}
  \item nodeA0 -> 1
  \end{itemize}

\item serviceB
  \begin{itemize}
  \item nodeA1 -> 1
  \end{itemize}

\end{itemize}

\subsection{stop containers}
\begin{itemize}
\item nodeA0c1
  \end{itemize}

\end{document}
